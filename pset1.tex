%%%%%%%%%%%%%%%%%%%%%%
% Preamble %%%%%%%%%%%
\documentclass[12pt,authoryear]{article}
\usepackage[utf8]{inputenc}
\usepackage[english]{babel}

% Page formatting
\usepackage[margin=1in]{geometry}
\setlength{\parskip}{0.85em}
\setlength\parindent{0pt}
\usepackage{setspace}
\onehalfspacing

% Section formatting
\usepackage{sectsty}
\sectionfont{\fontsize{12}{14}\selectfont}
\usepackage{titlesec}
\titlespacing\subsubsection{0pt}{5pt}{-2.5pt}
\titlespacing\subsection{0pt}{5pt}{-2.5pt}
\titlespacing\section{0pt}{5pt}{-2pt}
\titleformat*{\subsection}{\normalsize\bfseries}

% Custom section and subsection headers
\renewcommand\thesection{Problem \arabic{section}}
\renewcommand\thesubsection{\alph{subsection})}
\renewcommand\thesubsubsection{\alph{subsection}.\arabic{subsubsection})}

% Tables and Figures
\usepackage{multirow}
\usepackage{booktabs}
\usepackage{caption}
\usepackage{subcaption}
\usepackage{threeparttable}
\usepackage{float}
\usepackage{tikz}

% Appendix
\usepackage[toc,page]{appendix}

% Hyperlinks
\usepackage{url}
\usepackage[hyperfootnotes=false]{hyperref}
\hypersetup{
    hidelinks,
    linkbordercolor = {1 1 1},
    citebordercolor = {1 1 1},
    urlbordercolor = {1 1 1} 
}
\def\UrlBreaks{\do\/\do-}

% Math
\usepackage{amsmath}    
\usepackage{mathtools}
\usepackage{bbm}
\usepackage{array}

% References
\usepackage{natbib}
\usepackage{apalike}
\let\OLDthebibliography\thebibliography
\renewcommand\thebibliography[1]{
  \OLDthebibliography{#1}
  \setlength{\parskip}{0pt}
  \setlength{\itemsep}{0pt plus 0.3ex}
}
\newcommand{\citePos}[1]{\citeauthor{#1}'s (\citeyear{#1})}

% Code
\usepackage{listings}
\lstset{frame=trBL,
  frameround=fttt,
  language=R,
  breakindent = 0pt,
  xleftmargin = 5mm,
  framexleftmargin=0mm,
  basicstyle={\footnotesize\ttfamily},
  commentstyle = {\small\ttfamily\color{gray}},
  numbers=left, 
  stepnumber=1, 
  numberstyle=\scriptsize\color{gray}
}

%%%%%%%%%%%%%%%%%%%%%%
% Main Document %%%%%%
\begin{document}

%%%%%%%%%%%%%%%%%%%%%%
% Document header %%%%
{\LARGE \centering ECON 21020 -- Problem Set 1\par}
{\vspace{-1em} \large \centering Alex Ding \par}
{\centering \vspace{-1em} \today \par }

%%%%%%%%%%%%%%%%%%%%%%
% Problem 1 %%%%%%%%%%
\section{}
Table 1 gives the joint probability mass function between employment status and college degree in the US working-age population.
\begin{center}
Table 1: Joint Probability Mass Function
\end{center}

\begin{table}[h!]
\centering
 \begin{tabular}{||c c c||} 
 \hline
 & Unemployed (Y = 0) & Employed (Y = 1)  \\ [0.5ex] 
 \hline\hline
 Non-college grads (X = 0) & 0.026 & 0.576 \\ 
 College grads (X = 1) & 0.009 & 0.389 \\ [1ex] 
 \hline
 \end{tabular}
\end{table}
 a) An economic interpretation of the statement P(Y = 1, X = 1) = 0.389 is that 0.389 of the working-age population in the U.S is a college grad and employed.
 
 b) The unconditional probability of being employed P(Y = 1) is 0.576 + 0.389 = 0.965. 
%%%%%%%%%%%%%%%%%%%%%%
% Problem 2 %%%%%%%%%%
\newpage
\section{}

This section provides some simple math expressions using Example 2.21 from \citet{wasserman2003all}.

Let $(X, Y)$ be random variables with density 
\begin{align}\label{pset1:eq_density}
    f(x, y) = \begin{cases} % Cases are used for the curly bracket
        x + y, & \quad \text{if } x, y \in [0, 1] \\
        0, & \quad \text{otherwise.}
    \end{cases}
\end{align}

To verify that \eqref{pset1:eq_density} is a density, note that
\begin{align}
    \begin{aligned} % aligned assures that there is a single equation number associated with this derivation
        \int_{-\infty}^\infty\int_{-\infty}^\infty f(x, y) dxdy & = \int_0^1\int_0^1 (x + y) dxdy \\
        & = \int_0^1\left(\int_0^1 x dx\right)dy + \int_0^1\left(\int_0^1 y dy\right)dx \\
        & = \int_0^1\frac{1}{2}dy + \int_0^1\frac{1}{2}dx \\
        & = \frac{1}{2} + \frac{1}{2}\\
        & = 1,
    \end{aligned}
\end{align}
which is the desired result.

\href{https://oeis.org/wiki/List_of_LaTeX_mathematical_symbols}{\underline{This website}} is a great resource for math symbols (e.g., Greek letters such as $\alpha, \Sigma$ or $\varepsilon$).

%%%%%%%%%%%%%%%%%%%%%%
% Problem 3 %%%%%%%%%%
\newpage
\section{}

This section contains a simple table for presenting, for example, summary statistics or regression results. Table \ref{pset1:tab_estres} provides (entirely made-up) results of linear regression (Panel A) and two-stage least squares (Panel B) on the returns to education. 

\begin{table}[!htbp] \small 
  \centering 
  \begin{threeparttable}
  \caption{Estimation Results}\label{pset1:tab_estres}
\begin{tabular}{llcc} \toprule \midrule 
      &       & Employed & $\log$-Income  \\ 
      &       & (1) & (2) \\ \midrule
\multicolumn{3}{l}{\textit{Panel A:  OLS}} \\
      & High School & 0.262 & 0.290 \\
      &       & (0.063) & (0.120) \\
      & College Degree & 0.345 & 0.400\\
      &       & (0.087) & (0.091) \\
      &       &          & \\
\multicolumn{3}{l}{\textit{Panel B:  TSLS}} \\
      & High School & 0.0797 & 0.22 \\
      &       & (0.0281) & (0.045) \\
      & College Degree & 0.121 & 0.321\\
      &       & (0.030) & (0.123) \\
      &       &          & \\
      &  Observations & 320 & 282 \\ \midrule \bottomrule
\end{tabular}%
  \begin{tablenotes}[para,flushleft]
  \footnotesize \item \textit{Notes.} Table notes go here.
  \end{tablenotes}
    \end{threeparttable}
\end{table}

There are many tools for generating \LaTeX \ tables, from simple websites (e.g., \href{https://www.tablesgenerator.com/}{\underline{tablesgenerator.com}}) to packages for your favorite programming language (e.g., \href{https://cran.r-project.org/web/packages/xtable/index.html}{\underline{\texttt{xtable}}}). To get started, I recommend using the excel-plug in \href{https://ctan.org/thttps://cran.r-project.org/web/packages/xtable/index.htmlex-archive/support/excel2latex?lang=en}{\underline{\texttt{excel2latex}}}. 

%%%%%%%%%%%%%%%%%%%%%%
% Problem 4 %%%%%%%%%%
\newpage
\section{}

This section contains a simple figure with two panels.

\begin{figure}[!htbp]
    \centering
    \caption{Pictures of Pets}
     \begin{subfigure}[b]{0.285\textwidth}
     \centering
     \includegraphics[width=\textwidth]{Figures/pset1/dog.jpeg}
     \caption{A dog}
     \end{subfigure} \hspace{1em}
     \begin{subfigure}[b]{0.4\textwidth}
     \centering
     \includegraphics[width=\textwidth]{Figures/pset1/cat.jpeg}
     \caption{A cat}
     \end{subfigure} 
     \vskip0.7em
\begin{minipage}{1\textwidth} 
{\footnotesize \begin{spacing}{1} \textit{Notes.} Figure notes would go here.\end{spacing}} 
\end{minipage}
\end{figure}

%%%%%%%%%%%%%%%%%%%%%%
% References %%%%%%%%%
\newpage
\interlinepenalty=10000
\addcontentsline{toc}{section}{References}
\bibliographystyle{apalike}
\bibliography{biblio}
\interlinepenalty=10

%%%%%%%%%%%%%%%%%%%%%%
% Appendix %%%%%%%%%%%
\newpage
\appendix

\section{Code}

This appendix contains the code used for the analysis. 

\begin{lstlisting}[language=R, basicstyle=\footnotesize\ttfamily]
# Insert code here (if applicable).
x <- FALSE
if (x) {
    make_a_joke()
}#IF
\end{lstlisting}


\end{document}