%%%%%%%%%%%%%%%%%%%%%%
% Preamble %%%%%%%%%%%
\documentclass[12pt,authoryear]{article}
\usepackage[utf8]{inputenc}
\usepackage[english]{babel}

% Page formatting
\usepackage[margin=1in]{geometry}
\setlength{\parskip}{0.85em}
\setlength\parindent{0pt}
\usepackage{setspace}
\onehalfspacing

% Section formatting
\usepackage{sectsty}
\sectionfont{\fontsize{12}{14}\selectfont}
\usepackage{titlesec}
\titlespacing\subsubsection{0pt}{5pt}{-2.5pt}
\titlespacing\subsection{0pt}{5pt}{-2.5pt}
\titlespacing\section{0pt}{5pt}{-2pt}
\titleformat*{\subsection}{\normalsize\bfseries}

% Custom section and subsection headers
\renewcommand\thesection{Problem \arabic{section}}
\renewcommand\thesubsection{\alph{subsection})}
\renewcommand\thesubsubsection{\alph{subsection}.\arabic{subsubsection})}

% Tables and Figures
\usepackage{multirow}
\usepackage{booktabs}
\usepackage{caption}
\usepackage{subcaption}
\usepackage{threeparttable}
\usepackage{float}
\usepackage{tikz}

% Appendix
\usepackage[toc,page]{appendix}

% Hyperlinks
\usepackage{url}
\usepackage[hyperfootnotes=false]{hyperref}
\hypersetup{
    hidelinks,
    linkbordercolor = {1 1 1},
    citebordercolor = {1 1 1},
    urlbordercolor = {1 1 1} 
}
\def\UrlBreaks{\do\/\do-}

% Math
\usepackage{amsmath}    
\usepackage{mathtools}
\usepackage{bbm}
\usepackage{array}

% References
\usepackage{natbib}
\usepackage{apalike}
\let\OLDthebibliography\thebibliography
\renewcommand\thebibliography[1]{
  \OLDthebibliography{#1}
  \setlength{\parskip}{0pt}
  \setlength{\itemsep}{0pt plus 0.3ex}
}
\newcommand{\citePos}[1]{\citeauthor{#1}'s (\citeyear{#1})}

% Code
\usepackage{listings}
\lstset{frame=trBL,
  frameround=fttt,
  language=R,
  breakindent = 0pt,
  xleftmargin = 5mm,
  framexleftmargin=0mm,
  basicstyle={\footnotesize\ttfamily},
  commentstyle = {\small\ttfamily\color{gray}},
  numbers=left, 
  stepnumber=1, 
  numberstyle=\scriptsize\color{gray}
}

%%%%%%%%%%%%%%%%%%%%%%
% Main Document %%%%%%
\begin{document}

%%%%%%%%%%%%%%%%%%%%%%
% Document header %%%%
{\LARGE \centering ECON 21020 -- Problem Set 2\par}
{\vspace{-1em} \large \centering Alex Ding \par}
{\centering \vspace{-1em} \today \par }

%%%%%%%%%%%%%%%%%%%%%%
% Problem 1 %%%%%%%%%%
\section{}
\textbf{1.1}

True. If E[UX] = 0, then E[U$\mid$X] = 0.

By the Law of Iterated Expectations:

E[UX] = E[E[U$\mid$X]X]

Since we know that E[UX] = 0, we know that E[U$\mid$X] = 0.

\textbf{1.2}

Here is a counterexample.
Let X be the discrete distribution that takes either 1 or 2 with equal probability.

Thus E[X] = 3/2, then 1/E[X] = 2/3.

E[1/X] = E[1/(3/2)] = 3/4.

Since 2/3 does not equal to 3/4, the answer is false.

%%%%%%%%%%%%%%%%%%%%%%
% Problem 2 %%%%%%%%%%

\section{}

\textbf{2.1}

The drivers can be correct because the sample distribution does not have to be normal.

%%%%%%%%%%%%%%%%%%%%%%
% Problem 3 %%%%%%%%%%
\section{}

\textbf{3.1}

%%%%%%%%%%%%%%%%%%%%%%
% Problem 4 %%%%%%%%%%
\section{}

\textbf{4.1} Show that E[Y$\mid$ X = 1] = E[XY]/E[X].

E[XY] = E[YX]

E[YX] = E[Y $\mid$ X=1] E[X]

E[YX] = E[E[YX $\mid$ X]]  = E[XE[Y $\mid$ X]

E[YX] = E[Y$\mid$ X=1] * 1 * P(X=1) + E[Y$\mid$ X=0] * 0 P[X=0]

E[Y $\mid$X=0] * 0 * P[X=0] = 0.

The indicator I = p, because X is Bernoulli distr, so E[X] = I * [X = 1] = P[X=1].

E[X] = P(X=1).

E[XY] = E[Y$\mid$ X = 1] * E[X]

Thus, E[Y$\mid$ X = 1] = E[XY]/E[X].

\textbf{4.2}

We will do a ratio $A_{n} / B_{n}$ to calculate the sample analogue.

E[XY] = $A_{n}$

E[X] = $B_{n}$

1/n $\sum^n_{i=1}$ $(X_{i})Y_{i}$ for $A_{n}$,

1/n $\sum^n_{i=1}$ ($X_{i}$) for $B_n$, 

The ratio $A_{n}/B_{n}$ is: 
1/n $\sum^n_{i=1}$ $(X_{i})Y_{i}$ / (1/n $\sum^n_{i=1}$ $X_{i}$) = the sample analogue $\hat{\mu}_{Y\mid 1}$.


\textbf{4.3} Show that for X: supp X = {0,1}, E[Y$\mid$ X] = XE[Y$\mid$X =1] + (1-X)E[Y$\mid$X = 0]  

Multiply both sides by X:

XE[Y$\mid$ X] = $X^{2}$ E[Y$\mid$X =1] + X(1-X)E[Y$\mid$X = 0]

XE[Y$\mid$ X] = $X^{2}$ E[Y$\mid$X =1] + XE[Y$\mid$X = 0] - $X^{2}$ E[Y$\mid$X = 0]

As $X^{2}$ E[Y$\mid$X = 0] - X E[Y$\mid$X = 0] = 0 because 
$X^{2}$ E[Y$\mid$X = 0] = X E[Y$\mid$X = 0]

Using the same logic, XE[Y$\mid$X] = $X^{2}$ E[Y$\mid$X =1] = XE[Y$\mid$X = 1].

\textbf{4.4}
Show that your estimator is unbiased conditional on $\sum_{i}$ $X_{i}$ $>$ 0.

\textbf{4.5} Show that $\hat{\mu}_{Y\mid1}$ $\xrightarrow{p}$ E[Y$\mid$X = 1]

Do 4 step process:

1.
Define:

$A_{n}$ is congruent to 1/n $\sum^n_{i=1} X_{i}Y_{i}$

$B_{n}$ is congruent to 1/n $\sum^n_{i=1} X_{i}$

2. Define g(a,b) = a/b

3. 

By WLLN, we get that $A_{n}$ $\xrightarrow{p}$ E[XY]

By WLLN, we get that $B_{n}$ $\xrightarrow{p}$ E[X]

4. By CMT:

$g(A_{n}, B_{n})$ $\xrightarrow{p}$ E[YX] / E[X] 

From 4.1, we know that E[YX] / E[X] = E[Y $\mid$ X=1].

Thus, $g(A_{n}, B_{n})$ $\xrightarrow{p}$ E[Y $\mid$ X=1].

$\forall E[X] \in R$ except E[X] = 0.

\textbf{4.6} Show that $\sqrt{n}$ ($\hat{\mu}_{Y\mid1}$ - E[Y$\mid$ X= 1]) = $\sqrt{n}$ (1/n$\sum^n_{i=1} X_{i}U_{i}$ / (1/n$\sum^n_{i=1} X_{i}$)).

We know that 
$\hat{\mu}_{Y\mid1}$ = $\sum^n_{i=1} (X_{i}$ * $Y_{i}$) / ($\sum^n_{i=1} X_{i}$).

Where $U_{i}$ is congruent to $Y_{i} - E[Y_{i} $\mid$ X_{i}]$.

First, 
$Y_{i}$ = $E[Y_{i}$ $\mid$ $X_{i}$] + $U_{i}$

Thus,
$\sqrt{n}$ ($\hat{\mu}_{Y\mid1}$ - E[Y$\mid$ X= 1]) = $\sqrt{n}$ (1/n$\sum^n_{i=1} (E[Y_{i} $ $\mid$ $X_{i}]$ + $U_{i}$) $X_{i}$ / (1/n$\sum^n_{i=1} X_{i}$)).

To elaborate within the numerator:

$(E[Y_{i}$ $\mid$ $X_{i}$] + $U_{i}$) $X_{i}$) =  $E[Y_{i} $ $\mid$ $X_{i}]$ * $X_{i}$ + ($U_{i}$) $X_{i}$ 

From part 4.3, we find that XE[Y$\mid$X] = XE[Y$\mid$X = 1].

Therefore, $\sum^n_{i=1}$ ($(E[Y_{i}$ $\mid$ $X_{i}$] + $U_{i}$) $X_{i}$)/ ($\sum^n_{i=1} X_{i}$) =  $E[Y_{i}$ $\mid$ $X_{i} = 1]$ * $X_{i}$ + $U_{i}$ * $X_{i}$ / ($\sum^n_{i=1} X_{i}$). 

$\sum^n_{i=1}$ ($(E[Y_{i}$ $\mid$ $X_{i}$] + $U_{i}$) $X_{i}$)/ ($\sum^n_{i=1} X_{i}$) = $\sum^n_{i=1}$ ($U_{i}$ $X_{i}$) / ($\sum^n_{i=1} X_{i}$) + $E[Y_{i}$ $\mid$ $X_{i} = 1]$  

$\hat{\mu}_{Y\mid1}$ - $E[Y_{i}$ $\mid$ $X_{i} = 1]$ = $\sum^n_{i=1}$ ($U_{i}$ $X_{i}$) / ($\sum^n_{i=1} X_{i}$)

We can expand on this, saying that 

$\sqrt{n}$ ($\hat{\mu}_{Y\mid1}$ - E[Y$\mid$ X= 1]) = $\sqrt{n}$ (1/n$\sum^n_{i=1} X_{i}U_{i}$ / (1/n$\sum^n_{i=1} X_{i}$)).

\textbf{4.7}

From part 4.6,

$\sqrt{n}$ $\hat{\mu}_{Y\mid1}$ - E[Y$\mid$ X = 1] = $\sqrt{n}$ (1/n$\sum^n_{i=1} X_{i}U_{i}$ / (1/n$\sum^n_{i=1} X_{i}$))

From Slutsky's Theorem:

If $A_{n}$ $\xrightarrow{p}$ A and $B_{n}$ $\xrightarrow{d}$ B, then $A_{n}$/$B_{n}$  $\xrightarrow{d}$ A/b.

1. $A_{n}$ is congruent to $\sqrt{n}$ 1/n$\sum^n_{i=1} X_{i}U_{i}$

 $B_{n}$ is congruent to 1/n $\sum^n_{i=1} X_{i}$

2. By CLT, $A_{n}$ = ($\sqrt{n}$ 1/n$\sum^n_{i=1} X_{i}U_{i}$ - E[UX]) $\xrightarrow{d}$ N(0, Var(UX)).

3. By WLLN, $B_{n}$ = 1/n $\sum^n_{i=1} X_{i}$ $\xrightarrow{p}$ E[X_{i}] = P(X_{i} = 1)

4. $A_{n}$/$B_{n}$ $\xrightarrow{d}$ 1/P(X=1) * N(0, Var(UX))

Need to find E[XU] = 0, and Var(XU) = Var(Y$\mid$X=1)P(X=1)/ $P(x=1)^2$

Function 1: 

$E[U^2 * X]$ = $E[E[U^2 X$ $\mid$ X]] = 0 = E[E[U$\mid$X]X]

= $E[U^2 * X]$

Function 2: Var ($\sum$UX /$\sum$(X))

Var(UX) = E[$U^2 X^2$] = E[$UX]^2$


= E[U^2 $\mid$ X = 1] P(X=1)

= E[(Y-E[Y$\mid$ $X])^2$ $\mid$ X=1] P(X=1)

= E[(Y-E[Y$\mid$X=1])^2 $\mid$X= 1] P(X=1)

= Var(Y$\mid$X=1)P(X=1)

Var(UX) = Var(Y$\mid$X=1)P(X=1)/ $P(x=1)^2$

Var(Y$\mid$X=1)P(X=1)/ $P(x=1)^2$ = Var(Y$\mid$X=1)/P(X=1))

Thus, we can conclude that $\sqrt{n}$ ($\hat{\mu}_{Y\mid1}$ - E[Y$\mid$ X = 1]) $\xrightarrow{d}$ N(0,Var(Y$\mid$X=1)/ P(C=1))

\textbf{4.8}

Var(Y$\mid$X=1) = E[$Y^2$ \mid$X=1] - (E[Y$\mid X = 1])^2$

The sample analogue $\hat{\sigma}^2_{Y\mid1}$ =
$\sum^n_{i=1} X_{i}Y_{i}^2$/ $\sum^n_{i=1} X_{i}$ - ($\sum^n_{i=1} (X_{i}Y_{i})$ / $\sum^n_{i=1} X_{i})^2$

The sample analogue $\hat{p}_{X}$ = 1/n $\sum^n_{i=1}I\{X_{i}=1\}

\textbf{4.9}

Show that $\hat{\sigma}^2_{Y\mid1}$ $\xrightarrow{p}$ ${\sigma}^2_{Y\mid1}$

Do 4 step process:

1.
Define:

$A_{n}$ is congruent to 1/n $\sum^n_{i=1} X_{i}Y_{i}^2$

$B_{n}$ is congruent to 1/n $\sum^n_{i=1} X_{i}$

2. Define g(a,b) = a/b

3. 

By WLLN, we get that $A_{n}$ $\xrightarrow{p}$ E[X$Y^2$]

By WLLN, we get that $B_{n}$ $\xrightarrow{p}$ E[X]

4. By CMT:

$g(A_{n}, B_{n})$ $\xrightarrow{p}$ E[$Y^2$X] / E[X] 

We know that ($\sum^n_{i=1} (X_{i}Y_{i})$ / $\sum^n_{i=1} X_{i})$ $\xrightarrow{p}$ $(E[XY]/E[X])$ from part 4.5

Thus,
$\sum^n_{i=1} X_{i}Y_{i}^2$/ $\sum^n_{i=1} X_{i}$ - ($\sum^n_{i=1} (X_{i}Y_{i})$ / $\sum^n_{i=1} X_{i})^2$ $\xrightarrow{p}$ E[$Y^2$X] / E[X] - $(E[XY]/E[X])^2$

= $\hat{\sigma}^2_{Y\mid1}$ $\xrightarrow{p}$ ${\sigma}^2_{Y\mid1}$

$\forall E[X] \in R$ except E[X] = 0.



\end{document}